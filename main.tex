\RequirePackage[l2tabu, orthodox]{nag}
\documentclass[scheme=chinese, heading = true, UTF8]{ctexart}

% 载入需要的宏包
% =========数学符号宏包=========
\usepackage{amssymb}
\usepackage{amsmath}

\usepackage{hyperref}
\usepackage{xcolor}

% =========设置页边距宏包=========
\usepackage[left=2.8cm,right=2.5cm,top=2.54cm,bottom=2.54cm]{geometry}

% ========排版代码的宏包=========
\usepackage{minted}

%%% Local Variables:
%%% mode: latex
%%% TeX-master:"../main.tex"
%%% End:

% 进行必要的设置
% 重定义强调字体的代码,解决默认强调字体是italic,此时中文会用楷体代替,
% 在此设置为加粗,注意需要使用etoolbox宏包
\makeatletter
\let\origemph\emph
\newcommand*\emphfont{\normalfont\bfseries}
\DeclareTextFontCommand\@textemph{\emphfont}
\newcommand\textem[1]{%
  \ifdefstrequal{\f@series}{\bfdefault}
    {\@textemph{\CTEXunderline{#1}}}
    {\@textemph{#1}}%
}
\RenewDocumentCommand\emph{s o m}{%
  \IfBooleanTF{#1}
    {\textem{#3}}
    {\IfNoValueTF{#2}
      {\textem{#3}\index{#3}}
      {\textem{#3}\index{#2}}%
     }%
}
\makeatother   

% ========设置标题的格式========
\ctexset{
  section = {
    format+ = \zihao{-4} \heiti \raggedright,
    name = {,、},
    number = \chinese{section},
    beforeskip = 1.0ex plus 0.2ex minus .2ex,
    afterskip = 1.0ex plus 0.2ex minus .2ex,
    aftername = \hspace{0pt}
  },
  subsection = {
    format+ = \zihao{5} \heiti \raggedright,
    % name={\thesubsection、},
    name = {,、},
    number = \arabic{subsection},
    beforeskip = 1.0ex plus 0.2ex minus .2ex,
    afterskip = 1.0ex plus 0.2ex minus .2ex,
    aftername = \hspace{0pt}
  }
}

% 设置minted宏包编排代码的参数及用于latex代码排版的简化命令
\definecolor{listinggray}{gray}{0.92}
\setminted{fontsize=\tiny, mathescape, breaklines=true, breakautoindent=false, autogobble}
\newmintinline{tex}{fontsize=\normalsize}
\newmintinline[texinlinett]{tex}{fontsize=\normalsize,escapeinside=||}
\newminted{tex}{bgcolor=listinggray, frame=lines}
\newminted[texcodett]{tex}{bgcolor=listinggray, frame=lines, escapeinside=||}
\newminted[shell]{sh}{autogobble,fontsize=\small,frame=lines}
\newmintedfile{tex}{bgcolor=listinggray, linenos=true, frame=lines}

% ========不需要页眉=======
\pagestyle{plain}

% 定义提醒字体
\newcommand{\alert}[1]{\textcolor{red}{\textbf{#1}}}

% 定义引号命令
\newcommand{\qtmark}[1]{``#1''}

% 定义专有名词
\newcommand{\csv}{\texttt{CSV}文件}
\newcommand{\ltab}{\LaTeX{}表格}

% ==============LaTeX命令排版命令========================
\newcommand\cs[1]{\texttt{\textbackslash#1}}
\newcommand\pkg[1]{\texttt{#1}\textsuperscript{PKG}}
\newcommand\env[1]{\texttt{#1}}
\newcommand{\note}[1]{{%
  \color{magenta}{\bfseries 注意:}\emph{#1}}}

% 参考文献
\bibliographystyle{plain}

%%% Local Variables:
%%% mode: latex
%%% TeX-master:"../main.tex"
%%% End:


% 导言区,可以在此引入必要的宏包
\usepackage{texboxie}
% 读取csv数据生成表格的宏包
\usepackage{csvsimple}
\usepackage{pgfplotstable}
\usepackage{datatool}
\usepackage{booktabs}
\usepackage{siunitx}

\sisetup{round-mode=places,
         table-number-alignment = center-decimal-marker
         }

% 彩色表格
%\usepackage[table]{xcolor}
\usepackage{colortbl}

%%% 表格属性
\colorlet{tableheadcolor}{black!60}
\newcommand\tableheadfont{
  \sffamily\bfseries
  \slshape
  \color{white}
}

%\usepackage[skip=5pt]{caption}
\setlength{\abovecaptionskip}{0pt}% plus 2pt minus 1pt} % Chosen fairly arbitrarily
\setlength{\belowcaptionskip}{2pt plus 1pt minus 0pt} % Chosen fairly arbitrarily

\title{\Large \heiti 通过\csv 生成\ltab 的几种方法}
\author{\zihao{4} \fangsong 耿楠\\\small \songti 西北农林科技大学信息
  工程学院,陕西$\cdot$杨凌,712100}
\date{\today}

% 生成数据文件(必须置于导言区)
\begin{filecontents*}{db1.csv}
  姓名,性别,年龄
  张三,男,18
  李四,男,45
  马五,女,16
\end{filecontents*}
    
\begin{document}
  \maketitle

  \begin{abstract}
    \csv 称\emph{逗号分隔值}(Comma-Separated Values,CSV,有时也称
    为\emph{字符分隔值},因为分隔字符也可以不是逗号)文件,以纯文本形式
    存储表格数据(数字和文本)。纯文本意味着该文件是一个字符序列,不含必
    须像二进制数字那样被解读的数据。\csv 由任意数目的记录组成,记录间
    以某种换行符分隔;每条记录由字段组成,字段间的分隔符是其它字符或字
    符串,最常见的是逗号或制表符。通常,所有记录都有完全相同的字段序列。
    \csv 都是纯文本文件,可以使用记事本、Excel等软件进行生成或编辑,是
    一种比较方便的数据管理方式。在\LaTeX 中可以采用\pkg{csvsimple}、
    \pkg{pgfplotstable}、\pkg{datatool}、\pkg{csvtools}等宏包直接使用
    \csv 的数据生成\ltab 。
  \end{abstract}

  \section{准备\csv  数据}
  \csv 数据可以使用记事本、Excel等软件生成,也可以在导言区用
  \env{filecontents*}环境生成。
  \begin{center}
    \begin{minipage}[h]{0.6\linewidth}
      \begin{codeonly}
        \begin{filecontents*}{db1.csv}
          姓名,性别,年龄
          张三,男,18
          李四,男,45
          马五,女,16
        \end{filecontents*}
      \end{codeonly}
    \end{minipage}
  \end{center}
    
  该代码会在当前工作目录下生成``db1.csv''数据文件。

  \section{使用\pkg{csvsimple}宏包生成\ltab }
  \pkg{csvsimple}是一个用于处理\csv 数据的宏包,它采用
  了\pkg{pgfkeys}的\texttt{key-value}语法,是一个基于已有数据生成表格的
  轻量级工具包。在导言区使用\qtmark{\texinline{\usepackage{csvsimple}}}便可以使
  用该宏包提供的功能,详情请查阅其说
  明\url{http://www.ctan.org/pkg/csvsimple}或
  \url{https://github.com/T-F-S/csvsimple}。
  \subsection{简单方式生成表格}
  使用\pkg{csvsimple}宏包的最简单方式是直接\cs{csvautotabular}命令生成
  表格,其代码如下,生成的表格如表\ref{tab01}所示。
  \begin{center}
    \begin{minipage}[h]{0.7\linewidth}
      \begin{codeonly}
          \begin{table}[htb]
            \centering
            \caption{使用\cs{csvautotabular}命令生成表格\label{tab01}}
            \csvautotabular{db1.csv}
          \end{table}
      \end{codeonly}
    \end{minipage}
  \end{center}
  
  \begin{table}[htb]
    \centering
    \caption{使用\cs{csvautotabular}命令生成表格\label{tab01}}
    \csvautotabular{db1.csv}
  \end{table}
  
  \subsection{读入数据生成}
  为了能够更为灵活地控制生成的表格,可以使用\cs{csvreader}命令读入数据,
  并对表格属性进行必要地设置,其代码如下,生成的表格如表\ref{tab02}所示。
  \begin{center}
    \begin{minipage}[h]{0.75\linewidth}
      \begin{codeonly}
          \begin{table}[htb]
            \centering
            \caption{使用\cs{csvreader}命令生成表格\label{tab02}}
            \csvreader[tabular=|c|c|c|c|,% 列格式
                    table head=\hline & 姓名 & 性别 & 年龄\\ \hline,% 表头
                    late after line=\\\hline % 表格线
                    ]%
                    {db1.csv}% 数据文件
                    {姓名=\name,性别=\gender,年龄=\age}% 字段命名
                    {\thecsvrow & \name & \gender & \age}% 读入数据生成表格
          \end{table}     
      \end{codeonly}      
    \end{minipage}
  \end{center}
  

  \begin{table}[htb]
    \centering
    \caption{使用\cs{csvreader}命令生成表格\label{tab02}}
    \csvreader[tabular=|c|c|c|c|,% 列格式
             table head=\hline & 姓名 & 性别 & 年龄\\ \hline,% 表头
             late after line=\\\hline % 表格线
            ]%
            {db1.csv}% 数据文件
            {姓名=\name,性别=\gender,年龄=\age}% 字段命名
            {\thecsvrow & \name & \gender & \age}% 读入数据生成表格
  \end{table}

  \subsection{读入数据生成三线表}
  结合\pkg{booktabs}宏包,可以非常方便的用\cs{csvautobooktabular}直接
  生成三线表格,其代码如下,生成的表格如表\ref{tab03}所示。
  \begin{center}
    \begin{minipage}[h]{0.75\linewidth}
      \begin{codeonly}
          \begin{table}[htb]
            \centering
            \caption{使用\cs{csvautobooktabular}命令生成三线表\label{tab03}}
            \csvautobooktabular{db1.csv}% 数据文件
          \end{table}     
      \end{codeonly}      
    \end{minipage}
  \end{center}  

  \begin{table}[htb]
    \centering    
    \caption{使用\cs{csvautobooktabular}命令生成三线表\label{tab03}}
    \csvautobooktabular{db1.csv}% 数据文件
  \end{table}           

  如果再结合\pkg{siunitx}、\pkg{longtable}、\pkg{xcolor}等宏包,则可以对
  生成的表格进行更为细致的控制,有关细节,请查阅\pkg{csvsimple}宏包使
  用手册。

  \section{使用\pkg{pgfplotstable}宏包生成\ltab }
  \pkg{pgfplotstable}是一个用于处理\csv 数据的宏包,它为生成表格提供了
  丰富的设置命令。在导言区使用\qtmark{\texinline{\usepackage{pgfplotstable}}}便可以使
  用该宏包提供的功能,详情请查阅其说
  明\url{https://ctan.org/pkg/pgfplotstable}。
  \subsection{简单方式}
  使用\pkg{pgfplotstable}宏包的最简单方式是直
  接\cs{pgfplotstabletypeset}命令生成表格,利用该命令的可选参数,可以对
  最终生成表格的格式进行必要的设置,并且可以选择性的选取要输出的数据列
  或是对数据列的顺序进行调整。其示例代码如下,生成的表格如
  表\ref{tab04}所示\footnote{该例中使用了db2.csv数据,第1行是各列的名
    称,可以重新为各列命名。}。
  \begin{center}
    \begin{minipage}[h]{0.75\linewidth}
      \begin{codeonly}
        \begin{table}[htb]
          \centering    
          \caption{使用\cs{pgfplotstabletypeset}命令生成表格\label{tab04}}
          \pgfplotstabletypeset[
              col sep=comma,
              string type,
              columns/name/.style={column name=姓名, column type={|l}},
              columns/gender/.style={column name=性别, column type={|l}},
              columns/age/.style={column name=年龄, column type={|c|}},
              every head row/.style={before row=\hline,after row=\hline},
              every last row/.style={after row=\hline},
              ]{db2.csv}
        \end{table}
      \end{codeonly}
    \end{minipage}
  \end{center}

  \begin{table}[htb]
    \centering    
    \caption{使用\cs{pgfplotstabletypeset}命令生成表格\label{tab04}}
    \pgfplotstabletypeset[
        col sep=comma,
        string type,
        columns/name/.style={column name=姓名, column type={|l}},
        columns/gender/.style={column name=性别, column type={|l}},
        columns/age/.style={column name=年龄, column type={|c|}},
        every head row/.style={before row=\hline,after row=\hline},
        every last row/.style={after row=\hline},
        ]{db2.csv}
  \end{table}

  \subsection{结合\pkg{booktabs}生成三线表}
  结合\pkg{booktabs}宏包,通过\cs{pgfplotstabletypeset}命令的的可选参
  数,可以方便地实现三线表。其示例代码如下,生成的表格如
  表\ref{tab05}所示\footnote{该例中使用了db2.csv数据,第1行是各列的名
    称,该例中使用列号选择各列,并为各列重新命名。}。
  \begin{center}
    \begin{minipage}[h]{0.8\linewidth}
      \begin{codeonly}
        \begin{table}[htb]
          \centering    
          \caption{结合\pkg{booktabs}宏包生成三线表\label{tab05}}
          \pgfplotstabletypeset[
              col sep=comma,
              string type,
              columns/0/.style={column name=姓名, column type={l}},
              columns/1/.style={column name=性别, column type={l}},
              columns/2/.style={column name=年龄, column type={c}},
              every head row/.style={before row=\toprule,after row=\midrule},
              every last row/.style={after row=\bottomrule},
              ]{db2.csv}
        \end{table}
      \end{codeonly}
    \end{minipage}
  \end{center}

  \begin{table}[htb]
    \centering    
    \caption{结合\pkg{booktabs}宏包生成三线表\label{tab05}}
    \pgfplotstabletypeset[
        col sep=comma,
        string type,
        columns/0/.style={column name=姓名, column type={l}},
        columns/1/.style={column name=性别, column type={l}},
        columns/2/.style={column name=年龄, column type={c}},
        every head row/.style={before row=\toprule,after row=\midrule},
        every last row/.style={after row=\bottomrule},
        ]{db2.csv}
  \end{table}
      

  \subsection{结合\pkg{siunitx}控制数据显示精度}
  结合\pkg{siunitx}宏包,通过\cs{pgfplotstabletypeset}命令的的可选参
  数,可以方便地实现对数据显示精度的控制。其示例代码如下,生成的表格如
  表\ref{tab06}所示\footnote{该例中使用了db3.csv数据,第1行是各列的名
    称,注意数据中小数点不可省略。}。
  \begin{center}
    \begin{minipage}[h]{0.9\linewidth}
      \begin{codeonly}
        \begin{table}[htb]
          \centering    
          \caption{结合\pkg{siunitx}宏包控制数据显示精度\label{tab06}}
          \pgfplotstabletypeset[
            multicolumn names, 
            col sep=comma, 
            display columns/0/.style={
              column name=$Value 1$,
              column type={S[table-format = 3.0  ,round-precision=0]},string type}, 
            display columns/1/.style={
              column name=$Value 2$,
              column type={S[table-format = 3.2  ,round-precision=2]},
              string type},
            every head row/.style={
              before row={\toprule},
              after row={
                \si{\ampere} & \si{\volt}\\ 
                \midrule}
              },
            every last row/.style={after row=\bottomrule}, %
          ]{db3.csv} % 
      \end{table}
      \end{codeonly}
    \end{minipage}
  \end{center}

  \begin{table}[htb]
    \centering    
    \caption{结合\pkg{siunitx}宏包控制数据显示精度\label{tab06}}
    \pgfplotstabletypeset[
      multicolumn names, % 允许合并列名
      col sep=comma, % 逗号数据分割
      display columns/0/.style={
        column name=$Value 1$, % 列名称
        column type={S[table-format = 3.0  ,round-precision=0]},string type},  % 使用siunitx控制格式
      display columns/1/.style={
        column name=$Value 2$,
        column type={S[table-format = 3.2  ,round-precision=2]},
        string type},
      every head row/.style={
        before row={\toprule},
        after row={
          \si{\ampere} & \si{\volt}\\ % 数据单位,用 & 分割
          \midrule}
        },
      every last row/.style={after row=\bottomrule}, %
    ]{db3.csv} % filename/path to file
  \end{table}
  
  另外,也可以使用\pkg{multirow}宏包、\cs{multicolumn}命令实现表格的行
  列合并。如果再结合\pkg{longtable}、\pkg{array}、\pkg{colortbl}等宏包,则可以对
  生成的表格进行更为细致的控制,有关细节,请查阅\pkg{pgfplotstable}宏包使
  用手册。
        
  \section{使用\pkg{datatool}宏包}
  \pkg{datatool}是一个用于处理\csv 数据的宏包,使用它提供
  的\cs{DTLloaddb}结合\cs{DTLdisplaydb}命令或\cs{DTLforeach}循环,可以
  实现通过数据生成表格的操作。在导言区使
  用\qtmark{\texinline{\usepackage{datatool}}}便可以使用该宏包提供的功
  能,详情请查阅其说明\url{https://ctan.org/pkg/datatool}。
  \subsection{简单方式}
  可以通过\cs{DTLloaddb}命令载入\csv 数据,然后用\cs{DTLdisplaydb}命令
  生成表格。其示例代码如下,生成的表格如
  表\ref{tab07}所示\footnote{此处使用\cs{\DTLremoverow{mydb}{3}}命令删
    除了第3行数据。}。

  \begin{center}
    \begin{minipage}[h]{0.9\linewidth}
      \begin{codeonly}
        \DTLloaddb[keys={col1,col2,col3}]{mydb}{db1.csv}% 可以在之前任何位置载入数据
        \begin{table}[htb]
          \centering    
          \caption{\cs{DTLdisplaydb}命令生成表格\label{tab07}}          
          \DTLdisplaydb{mydb} % 
      \end{table}
      \end{codeonly}
    \end{minipage}
  \end{center}

  \DTLloaddb[keys={col1,col2,col3}]{mydb}{db1.csv}% 可以在之前任何位置载入数据
  \DTLremoverow{mydb}{3} % 删除第3行数据
  \begin{table}[htb]
    \centering
    \caption{\cs{DTLdisplaydb}命令生成表格\label{tab07}}
    \DTLdisplaydb{mydb} %
  \end{table}

  \subsection{使用\cs{DTLforeach}循环构建表格}
  可以通过\cs{DTLloaddb}命令载入\csv 数据,然后用\cs{DTLforeach}循环命令
  生成表格,此时,可以按普通表格的编写方式进行数据处理。其示例代码如下,生成的表格如
  表\ref{tab08}所示。
  \begin{center}
    \begin{minipage}[h]{0.7\linewidth}
      \begin{codeonly}
        \DTLloaddb{table}{db2.csv}% 可以在之前任何位置载入数据
        \begin{table}[htb]
          \centering    
          \caption{\cs{DTLforeach}循环命令生成表格\label{tab08}}
          \begin{tabular}{llc}
            \toprule
            姓名 & 年龄 & 性别 \tabularnewline
            \midrule
            \DTLforeach*{table}{\name=name, \gender=gender, \age=age}%
              {\DTLiffirstrow{}{\tabularnewline}%
              \name & \age & \gender}\\ % 数据列位置可任意
            \bottomrule
          \end{tabular}  
        \end{table}
      \end{codeonly}
    \end{minipage}
  \end{center}
  \DTLloaddb{table}{db2.csv}% 可以在之前任何位置载入数据
  \DTLremoverow{table}{2} % 删除第2行数据
  \begin{table}[htb]
    \centering    
    \caption{\cs{DTLforeach}循环命令生成表格\label{tab08}}
    \begin{tabular}{llc}
      \toprule
      姓名 & 年龄 & 性别 \tabularnewline
      \midrule
      \DTLforeach*{table}%
        {\name=name, \gender=gender, \age=age}%
        {\DTLiffirstrow{}{\tabularnewline}%
        \name & \age & \gender}\\ % 数据列位置可任意
      \bottomrule
    \end{tabular}  
  \end{table}
        
  如果再结合\pkg{siunitx}、\pkg{longtable}、\pkg{booktabs}、\pkg{array}、\pkg{colortbl}等宏包,则可以对
  生成的表格进行更为细致的控制,有关细节,请查阅\pkg{pgfplotstable}宏包使
  用手册。

  \section{结论}

  \LaTeX 排版技术经历了风风雨雨,已积累了大量的相关领域的宏包,借用这
  些宏包,可以大大减轻排版的工作量。发现这些宏包,善用这些宏包,就可以为
  我们的工作带来便利。
    
  
\end{document}

%%% Local Variables:
%%% mode: latex
%%% TeX-master: t
%%% End:
