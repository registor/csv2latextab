% 重定义强调字体的代码,解决默认强调字体是italic,此时中文会用楷体代替,
% 在此设置为加粗,注意需要使用etoolbox宏包
\makeatletter
\let\origemph\emph
\newcommand*\emphfont{\normalfont\bfseries}
\DeclareTextFontCommand\@textemph{\emphfont}
\newcommand\textem[1]{%
  \ifdefstrequal{\f@series}{\bfdefault}
    {\@textemph{\CTEXunderline{#1}}}
    {\@textemph{#1}}%
}
\RenewDocumentCommand\emph{s o m}{%
  \IfBooleanTF{#1}
    {\textem{#3}}
    {\IfNoValueTF{#2}
      {\textem{#3}\index{#3}}
      {\textem{#3}\index{#2}}%
     }%
}
\makeatother   

% ========设置标题的格式========
\ctexset{
  section = {
    format+ = \zihao{-4} \heiti \raggedright,
    name = {,、},
    number = \chinese{section},
    beforeskip = 1.0ex plus 0.2ex minus .2ex,
    afterskip = 1.0ex plus 0.2ex minus .2ex,
    aftername = \hspace{0pt}
  },
  subsection = {
    format+ = \zihao{5} \heiti \raggedright,
    % name={\thesubsection、},
    name = {,、},
    number = \arabic{subsection},
    beforeskip = 1.0ex plus 0.2ex minus .2ex,
    afterskip = 1.0ex plus 0.2ex minus .2ex,
    aftername = \hspace{0pt}
  }
}

% 设置minted宏包编排代码的参数及用于latex代码排版的简化命令
\definecolor{listinggray}{gray}{0.92}
\setminted{fontsize=\tiny, mathescape, breaklines=true, breakautoindent=false, autogobble}
\newmintinline{tex}{fontsize=\normalsize}
\newmintinline[texinlinett]{tex}{fontsize=\normalsize,escapeinside=||}
\newminted{tex}{bgcolor=listinggray, frame=lines}
\newminted[texcodett]{tex}{bgcolor=listinggray, frame=lines, escapeinside=||}
\newminted[shell]{sh}{autogobble,fontsize=\small,frame=lines}
\newmintedfile{tex}{bgcolor=listinggray, linenos=true, frame=lines}

% ========不需要页眉=======
\pagestyle{plain}

% 定义提醒字体
\newcommand{\alert}[1]{\textcolor{red}{\textbf{#1}}}

% 定义引号命令
\newcommand{\qtmark}[1]{``#1''}

% 定义专有名词
\newcommand{\csv}{\texttt{CSV}文件}
\newcommand{\ltab}{\LaTeX{}表格}

% ==============LaTeX命令排版命令========================
\newcommand\cs[1]{\texttt{\textbackslash#1}}
\newcommand\pkg[1]{\texttt{#1}\textsuperscript{PKG}}
\newcommand\env[1]{\texttt{#1}}
\newcommand{\note}[1]{{%
  \color{magenta}{\bfseries 注意:}\emph{#1}}}

% 参考文献
\bibliographystyle{plain}

%%% Local Variables:
%%% mode: latex
%%% TeX-master:"../main.tex"
%%% End:
